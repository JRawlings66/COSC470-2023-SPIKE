\documentclass[12pt]{article}

\usepackage[utf8]{inputenc}
%\usepackage{datatool} %database/graph/tables package

\begin{document}

%styling
%\usepackage[letterpaper, top=1.0in, bottom=1.0in, left=1.0in, right=1.0in, heightrounded]{geometry}
%line height
\renewcommand{\baselinestretch}{1.15} %line spacing
%paragraph indent
\setlength{\parindent}{0pt} %size of tab indent
\setlength{\parskip}{0.8em} %space between paragraphs

\title{Database Unit Tests}
\maketitle

\section{Null Testing}

\subsection{Summary}

Null testing is the simple procedure of attempting an insert into a table with all fields of the row being null. This will be done to every table that has a mandatory field, which will be all of them, and every single test should result in a failed insert. This test is also weak constraint checking, since null fields will never be within a constraint. When this test is ran, it will attempt to insert entries into each table with all null values.

\subsection{Expected Output}

We expect the database to reject insertions where every field is a null value.

We also expect the database to reject null inputs on fields with not-null constraints placed on them (primary/foreign keys, other fields based on business rules). 

%\subsubsection{Constraints}

%\subsubsection{Unit Tests}

%\subsection{BondDB}

%\subsection{CompanyDB}

%\subsection{CommodityDB}

%\subsection{notes}

%unit tests on connection script between collection/database
%write up two theoretical unit tests we could do
%unit tests on triggers
%tests for constraints??
%write a test trying to insert faulty data into the database

\end{document}