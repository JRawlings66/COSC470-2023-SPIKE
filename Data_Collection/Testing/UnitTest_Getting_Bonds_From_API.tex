%! Author = Dylan S, Vanessa D
%! Date = 9/27/2023

\documentclass{article}
\usepackage{geometry}
\usepackage{listings}
\usepackage{hyperref}

\geometry{a4paper, margin=1in}
\title{Unit Test Documentation: Bonds API Condig Load}
\author{Data Collection Team}
\date{\today}

\begin{document}
\maketitle


\section{Introduction}\label{sec:introduction}
This document provides documentation for a unit test that verifies the functionality of the API configuration loading is implemented in the bonds\_api\_query Python script.
The unit test is designed to ensure that the API call retrieves data successfully from the specified API endpoint using the provided JSON configuration.

\section{Test Objective}\label{sec:test-objective}
The objective of this unit test is to confirm that the API call to the specified URL with the provided parameters and API key results in a successful response (HTTP status code 200) and that the response data is properly parsed.

\section{Test Environment}\label{sec:test-environment}
The unit test is executed within the context of the Python script provided.
The script loads the JSON configuration, constructs the API query URL, makes the API call, and validates the response.

\section{Test Procedure}\label{sec:test-procedure}
The following steps outline the procedure for the unit test:

\begin{enumerate}
  \item \textbf{Load JSON Configuration}: The unit test begins by loading the JSON configuration file (\texttt{Bonds\_List.json}) that specifies the API URL, API key, date windows, and rate limit.

  \item \textbf{API Configuration Extraction}: The API configuration is extracted from the loaded JSON. For this test, we assume that there is a single API configuration in the JSON\@.

  \item \textbf{Construct API Query URL}: Using the extracted configuration, the unit test constructs the API query URL by replacing placeholders for start date, end date, and API key in the URL template.

  \item \textbf{Make API Call}: The unit test makes an HTTP GET request to the constructed API query URL using the \texttt{requests} library.

  \item \textbf{Validate HTTP Response}: The unit test checks the HTTP response status code to ensure it is 200 (OK), indicating a successful API call.
  Failure to receive a 200 status code will result in a test failure.

  \item \textbf{Parse JSON Response}: The unit test attempts to parse the JSON response received from the API call.
  Any failure in parsing will result in a test failure.

  \item \textbf{Check Data Availability}: The unit test checks whether the parsed JSON data is not empty.
  An empty response data will result in a test failure.

\end{enumerate}

\section{Expected Outcome}\label{sec:expected-outcome}
The expected outcome of this unit test is that the API call to the specified URL with the provided parameters and API key is successful.
The HTTP response status code is expected to be 200 (OK), and the parsed JSON response data should not be empty.

\section{Conclusion}\label{sec:conclusion}
This unit test verifies the functionality of the API call implemented in the Python script, ensuring that it can successfully retrieve data from the specified API endpoint.
A successful outcome confirms that the API integration is functioning as expected.

\end{document}
